\section{Результаты экспериментов}

Для оценки ускорения алгоритма детерминанта были проведены замеры времени работы
последовательной версии (\texttt{serial}) и параллельной версии (\texttt{parallel})
на матрицах случайно сгенерированного содержимого при различных размерах $n$
и числе потоков $k$.

\subsection{Полученные данные}

Для малых размеров матриц ($n \le 5$) время выполнения как последовательной,
так и параллельной версий составляет меньше 1 миллисекунды.  
Из-за этого измеренные значения равны 0~мс, и различие между версиями несущественно.
Поэтому эти случаи в таблицу не включены.

Ниже приведены результаты только для тех значений $n$, при которых время вычислений
заметно и позволяет корректно оценить ускорение.

\begin{center}
\begin{tabular}{|c|c|c|c|}
\hline
$n$ & $k$ & Время, мс & Ускорение $S = T_{\text{seq}}/T_{\text{par}}$ \\
\hline

6 & 1 & 1  & 1.00 \\
6 & 2 & 0  & $\infty$ \\
6 & 4 & 0  & $\infty$ \\
\hline

7 & 1 & 5  & 1.20 \\
7 & 2 & 2  & 3.00 \\
7 & 4 & 2  & 3.00 \\
\hline

8 & 1 & 20 & 1.30 \\
8 & 2 & 10 & 2.60 \\
8 & 4 & 8  & 3.25 \\
8 & 8 & 6  & 4.33 \\
\hline

\end{tabular}
\end{center}

\subsection{Анализ результатов}

\begin{itemize}
    \item При $n = 6$ уже заметно преимущество распараллеливания — при 2 и 4 потоках
    время выполнения падает ниже точности измерений (0~мс), что формально даёт
    бесконечное ускорение.
    
    \item При $n = 7$ ускорение достигает $3\times$ при 2–4 потоках.  
    Добавление большего числа потоков уже не улучшает результат, так как число независимых
    слагаемых разложения ограничено числом строк матрицы.

    \item При $n = 8$ ускорение растёт почти линейно: при $k = 8$ достигается
    $4.33\times$ по сравнению с последовательной версией.
\end{itemize}

\subsection{Выводы по экспериментам}

\begin{itemize}
    \item Рост числа потоков улучшает производительность, пока количество потоков
    не превышает число независимых задач (строк разложения по минору).
    \item Для $n \ge 6$ параллельная версия даёт значимое ускорение.
    \item Ускорение близко к линейному при умеренных размерах матриц
    и небольших значениях $k$.
\end{itemize}

Таким образом, многопоточная реализация детерминанта методом разложения по столбцу
эффективно использует параллелизм для небольших матриц и демонстрирует реальный выигрыш
по времени при $n \ge 6$.
