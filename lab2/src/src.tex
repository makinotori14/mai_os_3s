\section{Метод решения}

В работе используется определение детерминанта через разложение по столбцу
(формула Лапласа). Для квадратной матрицы $A$ порядка $n$ разложение по
первому столбцу имеет вид
\[
\det A = \sum_{i=0}^{n-1} (-1)^{i} a_{i0} \det M_{i0},
\]
где $M_{i0}$ --- минор, получающийся удалением $i$-й строки и нулевого столбца.

На основании этой формулы реализована рекурсивная функция \texttt{det},
которая:

\begin{itemize}
    \item для матриц размера $0$ возвращает $1$ (единичный элемент по умолчанию);
    \item для матриц $1 \times 1$ возвращает единственный элемент;
    \item для матриц $2 \times 2$ использует явную формулу
    $ad - bc$;
    \item для размерности $n \ge 3$ выполняет разложение по первому столбцу:
    \[
        \det A = \sum_{i=0}^{n-1} (-1)^{i} a_{i0} \det(\text{minor}(A, i, 0)).
    \]
\end{itemize}

Функция \texttt{minor} строит матрицу меньшего порядка, исключая заданную строку
и столбец. Знак $(-1)^i$ вычисляется вспомогательной функцией \texttt{sign}.

Для распараллеливания вычислений используется тот факт, что отдельные слагаемые
разложения по строкам независимы. В параллельной версии алгоритма:

\begin{itemize}
    \item для каждой строки первого столбца выделяется задача:
    вычислить сумму слагаемых разложения по подмножеству строк;
    \item индексы строк равномерно распределяются между потоками (по модулю числа потоков);
    \item каждый поток получает свою подмножество индексов строк и последовательно считает
    соответствующие слагаемые $\;(-1)^i a_{i0} \det M_{i0}$, накапливая частичный результат
    в отдельной целочисленной переменной;
    \item после завершения всех потоков главный поток суммирует частичные результаты,
    получая итоговый детерминант.
\end{itemize}

Для работы с потоками используется стандартная библиотека C++17:
\texttt{std::thread} и \texttt{std::vector}. Синхронизация между потоками не требуется,
так как каждый поток пишет только в свою ячейку массива частичных результатов, а общая
сумма вычисляется уже в главном потоке.

Максимальное количество потоков задаётся пользователем: сначала из стандартного ввода
считывается число потоков $k$, затем размер матрицы $n$. Число потоков ограничивается
\texttt{min(k, n)}, так как смысла в большем количестве потоков (чем число строк,
то есть число слагаемых в разложении) нет.

\section{Описание программы}

Исходный код разбит на следующие модули:

\begin{itemize}
    \item \texttt{det.hpp} --- заголовочный файл с объявлениями функций
    \texttt{sign}, \texttt{minor} и \texttt{det}.
    \item \texttt{det.cpp} --- реализация функции \texttt{sign}, вычисляющей
    $(-1)^i$ по номеру строки, функции \texttt{minor}, формирующей матрицу-минор,
    и рекурсивной функции \texttt{det}, вычисляющей детерминант методом разложения
    по первому столбцу.
    \item \texttt{serial.cpp} --- последовательная версия программы.
    Вводит число потоков \texttt{k} (для统一 интерфейса, хотя здесь не используется),
    затем размер матрицы \texttt{n} и элементы матрицы, после чего выводит результат
    вызова \texttt{det(a)}.
    \item \texttt{parallel.cpp} --- параллельная версия программы.
    Аналогично считывает \texttt{k} и \texttt{n}, затем матрицу.
    Формирует \texttt{k} векторов индексов строк, распределяя строки по схеме
    \texttt{rows[i \% k].push\_back(i)}. Для каждого потока создаётся задача
    \texttt{task}, которой передаются матрица, ссылка на переменную частичного
    результата и список строк; после завершения потоков главный поток суммирует
    элементы массива \texttt{part\_ans} и выводит итоговый детерминант.
    \item \texttt{tests/benchmark.cpp} --- модуль (при наличии), используемый
    для запуска последовательной и параллельной версий на случайных матрицах
    и измерения времени работы.
\end{itemize}

Таким образом, последовательная и параллельная версии используют одну и ту же
рекурсивную функцию \texttt{det}, отличаясь только схемой распределения слагаемых
разложения по потокам.
