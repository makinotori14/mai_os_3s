\section{Метод решения}

В работе используется определение детерминанта через разложение по первому столбцу
(формула Лапласа). Для квадратной матрицы $A$ порядка $n$:
\[
\det A = \sum_{i=0}^{n-1} (-1)^{i} a_{i0} \det M_{i0},
\]
где $M_{i0}$ — минор, получаемый удалением $i$-й строки и нулевого столбца.

Реализована рекурсивная функция \texttt{det\_single}, которая:

\begin{itemize}
    \item для $n = 0$ возвращает $1.0$ (нейтральный элемент умножения);
    \item для $n = 1$ возвращает $a_{00}$;
    \item для $n = 2$ вычисляет $a_{00}a_{11} - a_{01}a_{10}$;
    \item для $n \ge 3$ рекурсивно вычисляет сумму по строкам первого столбца.
\end{itemize}

Функция \texttt{minor} формирует подматрицу, исключая указанную строку и столбец.
Знак $(-1)^i$ вычисляется функцией \texttt{sign}.

Для распараллеливания используется независимость слагаемых разложения:
каждое слагаемое $\;(-1)^i a_{i0} \det M_{i0}$ может быть вычислено отдельно.
В параллельной версии:

\begin{itemize}
    \item индексы строк $0, 1, \dots, n-1$ равномерно распределяются между потоками;
    \item каждый поток получает диапазон строк $[start, end]$ и последовательно
    вычисляет соответствующие слагаемые, накапливая частичную сумму;
    \item используется многопоточная реализация \texttt{det\_parallel}, основанная
    на \texttt{pthread\_create} / \texttt{pthread\_join};
    \item результаты потоков собираются в главном потоке — явная синхронизация
    не требуется, так как каждый поток пишет только в свой объект \texttt{ThreadData}.
\end{itemize}

Число потоков ограничивается $\min(k, n)$, чтобы избежать создания лишних потоков
при $k > n$.

\section{Описание программы}

Исходный код разбит на следующие модули:

\begin{itemize}
    \item \texttt{det.hpp} — заголовочный файл с объявлениями \texttt{sign},
    \texttt{minor}, \texttt{det\_single} и \texttt{det\_parallel}.
    \item \texttt{det.cpp} — реализация вспомогательных функций и однопоточного
    вычисления \texttt{det\_single}.
    \item \texttt{det\_parallel.cpp} — реализация многопоточной функции
    \texttt{det\_parallel} на основе \texttt{pthread}.
    \item \texttt{main\_serial.cpp} — программа для запуска однопоточной версии:
    вводит размер матрицы и её элементы, выводит результат \texttt{det\_single}.
    \item \texttt{main\_parallel.cpp} — программа для запуска многопоточной версии:
    вводит число потоков и матрицу, вызывает \texttt{det\_parallel}.
    \item \texttt{tests/benchmark.cpp} — модуль на основе Google Test,
    автоматически генерирующий случайные матрицы, запускающий обе версии
    и измеряющий время выполнения и ускорение.
\end{itemize}

Обе программы (\texttt{serial} и \texttt{parallel}) используют общую логику
вычисления миноров и рекурсии, что гарантирует корректность сравнения.