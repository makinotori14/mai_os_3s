\section{Условие}

{\bfseries Цель работы:}

Приобретение практических навыков в:
\begin{itemize}
    \item управлении потоками в операционной системе;
    \item обеспечении синхронизации между потоками;
    \item оценке ускорения и эффективности параллельных алгоритмов.
\end{itemize}

{\bfseries Задание (общая постановка):}

Составить программу на языке C/C++, обрабатывающую данные в многопоточном режиме.  
При обработке использовать стандартные средства создания потоков операционной системы
(Windows/Unix). Ограничение максимального количества потоков, работающих в один момент
времени, должно задаваться ключом запуска программы.

Необходимо уметь продемонстрировать количество потоков, используемое программой,
с помощью стандартных средств операционной системы (например, \texttt{top}, \texttt{htop},
\texttt{ps}, диспетчер задач и т.\,п.).

{\bfseries Вариант 9.}

Рассчитать детерминант квадратной матрицы, используя определение детерминанта.  
Вычисления должны быть распараллелены по столбцам (или по группам слагаемых),
максимальное число одновременно работающих потоков задаётся параметром командной строки.

Требования к реализации:

\begin{itemize}
    \item реализовать последовательный алгоритм вычисления детерминанта по определению;
    \item реализовать многопоточную версию, в которой каждый поток вычисляет часть слагаемых
    (например, соответствующих одному столбцу или группе столбцов);
    \item обеспечить корректную синхронизацию при суммировании частичных результатов;
    \item обеспечить ограничение на максимально допустимое число потоков, задаваемое пользователем;
    \item провести серию экспериментов, сравнивая время работы последовательного и параллельного алгоритмов при различном размере матрицы и различном числе потоков.
\end{itemize}
