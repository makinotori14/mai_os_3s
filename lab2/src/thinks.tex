\section{Выводы}

В ходе выполнения лабораторной работы были приобретены практические навыки
управления потоками в операционной системе и организации параллельных вычислений
на языке C++ с использованием стандартной библиотеки \texttt{std::thread}.

Был реализован алгоритм вычисления детерминанта квадратной матрицы на основе
разложения по столбцу (формула Лапласа). Последовательная версия использует
рекурсивную функцию \texttt{det}, рассчитывающую миноры до базовых случаев
размерности $1$ и $2$.

На основе той же функции был построен многопоточный вариант: слагаемые разложения
по строкам первого столбца распределяются между потоками, каждый поток
вычисляет свою часть суммы и сохраняет результат в отдельной переменной.
Главный поток после завершения всех потоков суммирует частичные результаты
и получает итоговый детерминант. Такая схема не требует явной синхронизации,
так как отсутствует запись в общую разделяемую переменную из разных потоков.

В результате экспериментов было показано, что при разумном числе потоков
многопоточная версия даёт выигрыш по времени по сравнению с последовательной.
Однако при слишком большом числе потоков ускорение перестаёт расти из-за
накладных расходов на управление потоками и ограниченного числа независимых
задач (строк разложения). Кроме того, факториальная сложность алгоритма
разложения по минору делает его применимым только для матриц сравнительно
небольшого порядка.

Таким образом, лабораторная работа позволила на практике понять:
\begin{itemize}
    \item как устроены потоки исполнения в ОС и как ими управлять из C++;
    \item как распараллеливать независимые части вычислений (слагаемые разложения);
    \item какие ограничения накладывает как сама математическая сложность алгоритма,
    так и стоимость параллелизма.
\end{itemize}
