\section{Выводы}

В ходе выполнения лабораторной работы были приобретены практические навыки
работы с низкоуровневыми потоками операционной системы через POSIX-интерфейс
\texttt{pthread}.

Был реализован алгоритм вычисления детерминанта на основе разложения по
первому столбцу. Однопоточная версия (\texttt{det\_single}) использует рекурсивное
вычисление миноров. На её основе построена многопоточная версия
(\texttt{det\_parallel}), в которой каждому потоку поручается вычисление
части слагаемых суммы. Распределение строк выполняется равномерно,
а результаты собираются после завершения всех потоков.

Особенностью реализации является отказ от глобальных переменных и использование
структуры \texttt{ThreadData} для передачи параметров и результата — это обеспечивает
чистоту кода и предсказуемость поведения.

Эксперименты показали, что ускорение растёт с увеличением размера матрицы
и числа потоков, но быстро насыщается из-за факториальной сложности
базового алгоритма и накладных расходов на управление потоками.

Таким образом, работа позволила:
\begin{itemize}
    \item освоить создание, запуск и синхронизацию потоков через \texttt{pthread};
    \item научиться проектировать параллельные алгоритмы с независимыми подзадачами;
    \item оценить реальный выигрыш от параллелизма в условиях накладных расходов
    и ограничений алгоритмической структуры задачи.
\end{itemize}