\section{Условие}

{\bfseries Цель работы:}

\begin{itemize}
    \item приобретение практических навыков управления процессами ОС;
    \item освоение обмена данными между процессами посредством отображаемых файлов (memory-mapped files);
    \item изучение системных вызовов Unix: \texttt{fork}, \texttt{open}, \texttt{ftruncate}, \texttt{mmap}, \texttt{munmap}, \texttt{exec}, \texttt{waitpid}, а также базовых функций работы с файлами.
\end{itemize}

{\bfseries Задание:}

Родительский процесс создаёт дочерний процесс.  
Пользователь вводит имя файла, после чего родительский процесс открывает этот файл
и поочерёдно считывает из него целые числа.

Для организации обмена данными между процессами создаётся небольшой бинарный файл,
который отображается в адресное пространство обоих процессов с помощью системного
вызова \texttt{mmap}. В отображаемом файле хранится структура с текущим числом
и флагом состояния обмена.

Обмен организуется следующим образом:

\begin{itemize}
    \item родитель записывает очередное число во внутреннюю область отображаемого файла и
    выставляет флаг <<данные готовы>>;
    \item дочерний процесс читает число из общей памяти, проверяет его на простоту и
    в зависимости от результата:
    \begin{itemize}
        \item если число составное --- выводит его в стандартный поток вывода;
        \item если число отрицательное или простое --- выставляет флаг завершения
        обмена, после чего оба процесса корректно завершают работу.
    \end{itemize}
\end{itemize}

Количество чисел во входном файле может быть произвольным.

Родительский и дочерний процессы должны быть представлены разными программами.
Взаимодействие между ними осуществляется только через отображаемый файл.
