\section{Условие}

{\bfseries Цель работы:}

\begin{itemize}
    \item приобретение практических навыков управления процессами ОС;
    \item освоение обмена данными между процессами посредством каналов (pipe);
    \item изучение системных вызовов Unix: \texttt{fork}, \texttt{dup2}, \texttt{exec}, \texttt{read}, \texttt{write}, \texttt{waitpid}.
\end{itemize}

{\bfseries Задание:}

Родительский процесс создаёт дочерний процесс.  
Пользователь вводит имя файла, после чего родительский процесс открывает этот файл
и перенаправляет стандартный поток ввода дочернего процесса на этот файл.

Стандартный поток вывода дочернего процесса перенаправляется в канал (pipe1).
Родитель читает данные из pipe1 и выводит их в свой стандартный вывод.

Дочерняя программа читает из перенаправленного stdin последовательность чисел.
Для каждого числа выполняется проверка на простоту:

\begin{itemize}
    \item если число составное --- дочерний процесс выводит его в stdout (а значит, в pipe);
    \item если число отрицательное или простое --- оба процесса завершаются.
\end{itemize}

Количество чисел во входном файле может быть произвольным.

Родительский и дочерний процессы должны быть представлены разными программами.
