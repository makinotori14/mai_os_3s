\section{Выводы}

В ходе выполнения лабораторной работы были приобретены навыки работы с процессами
и взаимодействия между ними в ОС семейства Unix.  
Было реализовано и отлажено приложение, использующее системные вызовы:

\begin{itemize}
    \item \texttt{open} --- открытие файла;
    \item \texttt{pipe} --- создание канала для межпроцессного обмена;
    \item \texttt{fork} --- создание дочернего процесса;
    \item \texttt{dup2} --- перенаправление стандартных потоков;
    \item \texttt{execl} --- замена образа процесса исполняемым файлом;
    \item \texttt{read}/\texttt{write} --- обмен данными через pipe;
    \item \texttt{waitpid} --- корректное ожидание завершения дочернего процесса.
\end{itemize}

В результате работы была создана система из двух программ (родительской и дочерней),
обменивающихся данными через канал и корректно обрабатывающих системные ошибки.

Полученный опыт позволяет уверенно работать с базовыми механизмами межпроцессного
взаимодействия и потоков ввода-вывода в ОС Linux.
