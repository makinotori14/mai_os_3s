\section{Метод решения}

Для организации взаимодействия между процессами используется следующая схема:

\begin{enumerate}
    \item Родитель открывает файл, имя которого вводит пользователь.
    \item Родитель создаёт канал (pipe1) для передачи данных \emph{от дочернего процесса к родителю}.
    \item После вызова \texttt{fork()}:
    \begin{itemize}
        \item в дочернем процессе:
        \begin{itemize}
            \item вызов \texttt{dup2(filefd, STDIN\_FILENO)} перенаправляет stdin на файл;
            \item вызов \texttt{dup2(pipefd[1], STDOUT\_FILENO)} перенаправляет stdout в pipe;
            \item через \texttt{execl("./child", ...)} запускается программа \texttt{child}, заменяющая текущий образ процесса.
        \end{itemize}
        \item в родительском процессе:
        \begin{itemize}
            \item закрывается ненужный конец pipe (\texttt{pipefd[1]});
            \item родитель построчно читает данные из \texttt{pipefd[0]} функцией \texttt{read};
            \item затем выводит полученные строки в свой stdout.
        \end{itemize}
    \end{itemize}
    \item Родитель ожидает завершение дочернего процесса с помощью \texttt{waitpid}.
\end{enumerate}

Все ошибки системных вызовов обрабатываются через \texttt{perror} и корректное завершение работы.

\section{Описание дочернего процесса}

Программа \texttt{child.cpp}:

\begin{itemize}
    \item читает числа из стандартного ввода (который перенаправлен на файл);
    \item если число отрицательное --- завершает выполнение;
    \item если число простое --- завершает выполнение;
    \item если число составное --- выводит его в stdout (перенаправленный в pipe);
    \item продолжает работу, пока не встретит простое или отрицательное число.
\end{itemize}

\section{Описание родительского процесса}

Программа \texttt{parent.cpp}:

\begin{itemize}
    \item принимает имя файла от пользователя;
    \item открывает файл на чтение;
    \item создаёт pipe1;
    \item порождает дочерний процесс через \texttt{fork};
    \item перенаправляет потоки дочернего процесса и запускает исполняемый файл \texttt{child};
    \item читает результаты из pipe и выводит их в stdout;
    \item корректно завершает работу, ожидая дочерний процесс.
\end{itemize}
