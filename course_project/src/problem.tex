\section{Условие}

\textbf{Цель работы:}
\begin{itemize}
    \item изучить принципы межпроцессного взаимодействия в ОС Unix с использованием именованных каналов (FIFO);
    \item получить практические навыки работы с системными вызовами \texttt{mkfifo}, \texttt{open}, \texttt{read}, \texttt{write}, \texttt{close}, \texttt{unlink};
    \item освоить мультиплексирование ввода-вывода с помощью \texttt{poll}/\texttt{select};
    \item реализовать прототип клиент--серверного мессенджера с поддержкой личных и групповых сообщений.
\end{itemize}

\textbf{Задание:}

Вариант №22.

Необходимо реализовать программный прототип клиент--серверной системы мгновенного обмена сообщениями на C++ на базе Unix, используя только именованные каналы (FIFO).

Система должна поддерживать:
\begin{itemize}
    \item подключение клиента к серверу по логину;
    \item отправку личных сообщений по логину адресата;
    \item получение сообщений в реальном времени;
    \item поддержку групповых чатов (одно сообщение доставляется всем участникам группы);
    \item работу нескольких клиентов одновременно.
\end{itemize}

Требуемые операции (в терминах очередей/каналов сообщений):
\begin{itemize}
    \item \textbf{CreateQueue} --- создание канала сообщений (FIFO) для клиента или группы;
    \item \textbf{DeleteQueue} --- удаление канала сообщений;
    \item \textbf{ConnectToQueue} --- подключение к FIFO (получение файлового дескриптора);
    \item \textbf{Push (Send)} --- отправка сообщения (запись в FIFO);
    \item \textbf{Pop/Top (Receive)} --- получение сообщения (чтение из FIFO; \emph{Top} реализуется через буферизацию).
\end{itemize}

Ограничения:
\begin{itemize}
    \item вся коммуникация между сервером и клиентами должна идти через FIFO;
    \item для обслуживания нескольких источников ввода-вывода использовать \texttt{poll()} (или \texttt{select()}) либо \texttt{fork()}.
\end{itemize}

\pagebreak
