\section{Выводы}

В процессе выполнения курсовой работы была реализована клиент--серверная система мгновенного обмена сообщениями на основе именованных каналов FIFO в ОС Unix.

В результате:

\begin{itemize}
    \item освоены принципы межпроцессного взаимодействия через FIFO и особенности работы с \texttt{open/read/write} в неблокирующем режиме;
    \item получены практические навыки организации клиент--серверного протокола поверх текстовых команд;
    \item изучено мультиплексирование ввода-вывода с использованием \texttt{poll()} для обслуживания нескольких источников данных;
    \item реализованы операции CreateQueue/DeleteQueue/ConnectToQueue/Push/Pop (и логическая имитация Top через буферизацию);
    \item реализована поддержка групповых чатов с серверной рассылкой сообщений всем участникам.
\end{itemize}

Задача варианта №22 выполнена полностью: сервер и клиенты реализованы на C++ с применением системных вызовов Unix,
все взаимодействие между процессами происходит через FIFO.

\pagebreak
