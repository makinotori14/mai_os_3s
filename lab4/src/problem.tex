\section{Условие}

\textbf{Цель работы:}
\begin{itemize}
    \item изучение механизмов статической и динамической загрузки библиотек в ОС семейства Unix;
    \item получение навыков использования системных вызовов \texttt{dlopen}, \texttt{dlsym}, \texttt{dlclose};
    \item освоение техники разделения интерфейса (контрактов) и реализаций.
\end{itemize}

\textbf{Задание:}

Вариант №31.

Необходимо реализовать две функции:

\begin{enumerate}
    \item \textbf{Функция №6}: приближённое вычисление числа $e$.
    \begin{itemize}
        \item Первая библиотека: $E(x) = (1 + \frac{1}{x})^x$.
        \item Вторая библиотека: $E(x) = \sum\limits_{n=0}^{x} \frac{1}{n!}$.
    \end{itemize}

    \item \textbf{Функция №7}: вычисление площади прямоугольного треугольника.
    \[
        Area(a, b) = \frac{a \cdot b}{2}
    \]
\end{enumerate}

Должно быть реализовано:

\begin{itemize}
    \item файл с контрактами (\texttt{contracts.h}) c объявлением функций;
    \item две динамические библиотеки:
    \begin{itemize}
        \item \texttt{libimpl\_first.dylib} — первая реализация;
        \item \texttt{libimpl\_second.dylib} — вторая реализация;
    \end{itemize}
    \item две программы:
    \begin{itemize}
        \item \texttt{program1} — статическая линковка с первой библиотекой;
        \item \texttt{program2} — загрузка обеих библиотек через \texttt{dlopen} и вызов нужных функций в зависимости от команды.
    \end{itemize}
\end{itemize}

Команды:

\begin{itemize}
    \item \texttt{1 x} — вычислить $E(x)$ через библиотеку 1;
    \item \texttt{2 a b} — вычислить $Area(a,b)$ через библиотеку 2;
    \item \texttt{0} — переключение реализаций \textbf{не требуется}.
\end{itemize}

\pagebreak
