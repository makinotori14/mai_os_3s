\section{Метод решения}

В работе используются две динамические библиотеки, содержащие разные реализации одинаковых функций. Объявления функций вынесены в общий заголовочный файл контракта:

\begin{verbatim}
extern "C" double E(int x);
extern "C" double Area(double a, double b);
\end{verbatim}

Общий механизм:

\begin{itemize}
    \item \textbf{program1} компонуется со \texttt{libimpl\_first.dylib} на этапе сборки. 
          Все вызовы функций обращаются к первой реализации.
    \item \textbf{program2} загружает обе библиотеки вручную через:
    \begin{itemize}
        \item \texttt{dlopen(path, RTLD\_LAZY)};
        \item \texttt{dlsym(handle, "E")};
        \item \texttt{dlsym(handle, "Area")};
        \item \texttt{dlclose(handle)}.
    \end{itemize}
    \item В зависимости от команды вызывается функция \texttt{E} или \texttt{Area} из нужной библиотеки.
\end{itemize}

\subsection{Структура проекта}

\begin{verbatim}
lab4/
├── CMakeLists.txt
├── include/
│   └── contracts.h
└── src/
    ├── impl1.cpp
    ├── impl2.cpp
    ├── program1.cpp
    └── program2.cpp
\end{verbatim}

\subsection{Реализация библиотек}

Первая библиотека содержит:
\begin{itemize}
    \item вычисление $e$ через $(1 + 1/x)^x$;
    \item вычисление площади треугольника как $ab/2$.
\end{itemize}

Вторая библиотека содержит:
\begin{itemize}
    \item вычисление $e$ через сумму $1/n!$ до заданного предела;
    \item вычисление площади треугольника как $ab/2$ (альтернативная реализация допускается).
\end{itemize}

\subsection{Работа program1}

\begin{itemize}
    \item принимает команды \texttt{1 x}, \texttt{2 a b};
    \item вызывает функции из первой библиотеки;
    \item выводит результат.
\end{itemize}

\subsection{Работа program2}

\begin{itemize}
    \item вручную загружает первую и вторую библиотеки;
    \item получает адреса функций через \texttt{dlsym};
    \item по команде \texttt{1} вызывает функцию $E$ из \texttt{libimpl\_first};
    \item по команде \texttt{2} вызывает $Area$ из \texttt{libimpl\_second};
    \item завершает работу, закрывая обе библиотеки.
\end{itemize}

\pagebreak
